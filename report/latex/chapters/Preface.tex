\section{Preface}\label{sec:preface}

\subsection{Brief presentation of the protocol}
The \textbf{MeltyFi protocol} presented in the report proposes an innovative solution for \textbf{peer-to-pool lending and borrowing with NFT collateral}. Thanks to its structure and \textbf{independence from off-chain factors} such as the floor price of NFTs, MeltyFi allows borrowers to easily obtain loans \textbf{without the risk of involuntary liquidation of the NFT}, allowing them to obtain liquidity without risking the loss of their NFT. Additionally, the MeltyFi protocol allows lenders to use their capital to \textbf{provide liquidity for loans through a lottery system}. For this use of capital, lenders are obviously rewarded, and in the event that the loan they funded is not repaid, they have the opportunity to win the NFT used as collateral proportional to the capital they provided for that loan. If the loan is repaid, the lenders are obviously returned the capital they invested. Those who also repay a loan are rewarded with an amount equal to the interest paid to the protocol. 
\\
\indent In summary, MeltyFi guarantees for all users an easy access to a decentralized system of lending and borrowing with NTFs collateral, independent of external factors, and encourages users to provide liquidity and repay loans in order to be rewarded for their correct behavior in proportion to their personal capital invested in the protocol. MeltyFi represents an effective solution for \textbf{"making the illiquid liquid"} and offers \textbf{benefits to both borrowers and lenders}.

\subsection{Outline of the report}
\begin{itemize}
    \item \textbf{Background}: an introduction to all concept upon which MeltyFi is based is given (\autoref{sec:background}).
    \item \textbf{Presentation of the context}: contains the aim of the protocol, use cases and a neat explanation (\autoref{sec:presentationOfTheContext}).
    \item \textbf{Software Architecture}: this section describes the software architecture including the tools used, an overview of the protocol components, and a detailed examination of each component, as well as reflections on design choices (\autoref{sec:softwareArchitecture}).
    \item \textbf{Implementation}: here the frontend and the logics behind it are presented (\autoref{sec:implementation}).
    \item \textbf{Know issues and limitations}: limitations of the protocol are discussed in this section (\autoref{sec:KnownIssuesAndLimitations}).
\end{itemize}

\subsection{Team members and main responsibilities}
%autors: Vincenzo, Benigno, Andrea
%TODO
\begin{itemize}
    \item \textbf{Vincenzo Imperati}: Project manager, back-end developer, [project, web, report, brand] designer.
    \item \textbf{Benigno Ansanelli}: Developing and maintaining the frontend user interface, with a focus on lottery-related functionality. Preparing and delivering reports and presentations on lottery-related topics.
    \item \textbf{Andrea Princic}: Developing and maintaining the frontend user interface, with a focus on profile-related functionality. Designing use cases and sequence diagrams.
\end{itemize}


