\section{Conclusions}\label{sec:conclusions}
We presented MeltyFi, a new type of lending and borrowing protocol based on the peer-to-pool design, in which the loan collateral is NFT and the funds are raised through a lottery ticket foundraising mechanism. MeltyFi's design allows borrowers not to risk seeing their NFT liquidated unexpectedly because there is no dependence on external factors such as floor price. Borrowers also have full control over how much they wish to borrow (no predetermined loan-to-value, it is simply decided by the market) and the expire date of the loan. It will then be up to the lenders to finance the borrowers' loan requests. In this the lenders are motivated since they can choose which loan to finance, plus through the mechanism of expire lotteries the lender is incentivized to finance the borrowers because the lottery will be concluded at expire dates and not when the tickets are all sold. This still spurs people to buy tickets, facilitate financing for borrowers, and increase earnings for lenders in case of cancelled lotteries. 
\\
\indent In short, we have succinctly summarized the main features of the MeltyFi protocol and pointed out some win-win mechanisms for all users. In fact, with the proposed design, many issues present in platforms with the same intent as meltyfi are simply not there. In addition to not presenting many problems that plague other platforms, MeltyFi incentivizes lending and borrowing in a healthy way by minimizing risk and maximizing benefits for all users participating in the protocol. 
\\
\indent MeltyFi \textbf{melts finance} in a really elegant way and manages to turn even really very illiquid and volatile assets like NFTs into liquid assets.