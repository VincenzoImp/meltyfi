\section{Implementation}\label{sec:implementation}
\subsection{Tools used}
\paragraph{React} is a JavaScript library for building user interfaces. It is useful because it allows developers to build web applications with a component-based architecture, which makes it easy to manage the state and logic of different parts of the application separately. React also uses a virtual DOM, which optimizes the performance of updates and improves the overall speed of the application. React also has a large and supportive community and a wide range of available resources and third-party libraries, making it a popular choice for developers building web applications.\cite{react}
\paragraph{React-Bootstrap} is a library of pre-built components for React that allow developers to easily add Bootstrap styling to their projects. The use of pre-built components can help to improve the consistency and maintainability of the codebase. \cite{reactbootstrap}
\paragraph{Ethers.js} is a JavaScript library for interacting with the Ethereum blockchain. It provides a simple and easy-to-use interface for developers to interact with smart contracts, send transactions, and manage their digital assets on the Ethereum blockchain. Ethers.js is built on top of the Ethereum JSON-RPC API and supports both the web3.js and Web3.eth APIs. It is compatible with both Node.js and browser environments.
Ethers.js is useful for developers who want to build decentralized applications (DApps) on the Ethereum blockchain. It abstracts away the complexities of working directly with the Ethereum JSON-RPC API, making it easy for developers to interact with smart contracts, send transactions, and manage digital assets. Additionally, Ethers.js has a lot of functionalities that are not present in web3.js, such as contract events, contract transactions, and contract deployment. \cite{ethersJS}
\paragraph{thirdweb} is a development framework that allows you to build web3 functionality into your applications. \cite{thirdweb}
\paragraph{MetaMask} is a browser extension and cryptocurrency wallet that allows users to easily interact with decentralized applications (DApps) on the Ethereum blockchain. 
It acts as a bridge between the user's browser and the Ethereum network, providing users with a secure and user-friendly way to manage their cryptocurrency and interact with DApps. MetaMask is useful because it eliminates the need for users to run a full Ethereum node, making it easy for anyone to access and interact with the Ethereum blockchain. Additionally, it provides a simple and secure way for users to manage their cryptocurrency and interact with dApps, making it a powerful tool for users and developers alike. \cite{metamask}
\paragraph{OpenSea Testnets API} is used for browsing non-fungible assets on the Ethereum Goerli test network. We used it for retrieving all the asset for a certain address. \cite{testnetsAPI}

\subsection{MeltyFi.NFT DApp}
In this section, we will delve into the research and development process that led to the implementation of specific solutions for the frontend application of MeltyFi. We will examine the difficulties encountered during the process and how the tools and technologies cited above were utilized to overcome these challenges. The purpose of this analysis is to provide a comprehensive understanding of the decision-making process and the rationale behind the chosen solutions.

\subsubsection{Home page and website fruition}
The placement of the protocol description on the homepage serves as an effective introduction for users who may be unfamiliar with the technology. 
This approach allows them to quickly grasp the purpose and functionality of the protocol.
As the protocol becomes more widely adopted, this information may be relocated to an "Info" section, accompanied by tutorials to enhance the user's understanding and utilization of the DApp.
As best practice we inserted a navbar that linked to the various pages, so that in each point of the DApp the user could easily move around the website. 
At the bottom we inserted a footer that contains useful addresses such as MeltyFi. 

A fundamental thing is that thirdweb has a standard implementation of the login button that is really nice, and well integrated with MetaMask.

\subsubsection{Lotteries page}
The layout of this page features two tabs, "Browse Lotteries" and "Create Lottery". In order to access either of these tabs, it is necessary to connect a wallet. In principle, it is only necessary to have Metamask installed to view active lotteries, however, we opted to log in with a MetaMask account at the beginning of the session, rather than when purchasing tickets, as this will facilitate a faster buying process (crucial in case of a quick lottery).
\paragraph{Browse Lotteries tab} Allows users to view lotteries created by other users. In designing this feature, we conducted research on other lottery websites and aimed for the best user experience. This led us to the use of cards for each lottery. Since they feature an NFT as a prize, the actual image of the NFT is vital, as users are highly influenced by what they see when purchasing NFTs. In addition, the card displays relevant information about the lottery, such as the date, number of tickets sold, and the price. The only button present on each card allows users to purchase tickets for that lottery.\\
The recurring principle for the front end development was to optimize the user experience as much as possible. To achieve this goal, various subtle enhancements were implemented. For instance, the date is displayed according to the browser's language settings. Additionally, prices that are too low to be significant are not displayed, and an indicator of the number of tickets sold is provided to give users an indication of the popularity of a given lottery.
\\
\indent When a user decides to purchase a lottery ticket, they enter into a modal view, where the background is blurred to focus attention on the information for that lottery. This view includes a link to the user contract and another link to the NFT contract, allowing users to conveniently decide if the owner and contract are worth trusting. The Wonkabar price is also displayed no matter how small, and a selector allows users to choose the number of tickets to purchase, with the final price automatically computed. The recurrent idea is to prevent users from making mistakes in order to avoid the user (and the chain) failing transactions. The user can insert only positive integers numbers for the tickets. Additionally, the protocol forbids buying more than 25 percent of the tickets, so the user can't insert a number of tickets that makes this property false.
The front end prevents many errors that could occur, but in the event of problems, an error message will be displayed.
When the user clicks on \href{https://goerli.etherscan.io/tx/0xa8017555c918384405794472eb8559144dc563cc427703b9cb5e310c4aab6222}{buyWonkaBar}, a MetaMask form will appear, and the modal view will close once the transaction is completed (unless there's an error).
Another way in which the front-end helps the backend is in the case of the oracles failure (that probably due to insufficient funds, don't declare lotteries as closed), so the frontend filters the active lotteries given by the backend adding also a filter for lottery with an expiry date that is expired.
In one case, the fronted actually modified the backend: at the beginning there were many calls to know the information about a lottery. We realized that this really slowed the frontend, so we instead did only one call that returned all the information about a lottery. The loading speed of the frontend was one of the reason for adopting Goerli, a fairly fast testnet.
\paragraph{Create Lottery tab} Allows users to view and select from their available ERC721 NFTs to use as collateral for a loan. To retrieve a list of NFTs associated with a given address, the OpenSea API was utilized, as there was no existing method for easily retrieving NFTs using Ethers library, we reluctantly made this choice as it added yet another dependency. We filtered ERC721 type NFTs, the only that could be selected as collateral.
The card displayed for each NFT in the "Create Lottery" tab includes the collection name, token ID, and a button to initiate the lottery creation process. Once the user click on \href{https://goerli.etherscan.io/tx/0xe68f3f68b00dce4c299d0205dfbc72c302ae4d5e089b16d3c024755db70ffdf3}{createLottery}, they are presented with a modal view to input the value of each lottery ticket in Wei. The choice to use \$Wei was made as it allows for more flexibility in selecting values, particularly with limited resources. The option to use \$ETH in the future cannot be excluded. A date picker is also provided for the user to select the expiration date of the lottery, rather than requiring manual input of seconds. The total revenue from the sale of all lottery tickets is also displayed for the user's reference. To create a lottery, the user must first approve the transfer of the selected NFT to the MeltyFi protocol's address. The frontend first waits for the approval transaction (that authorizes the transfer of the NFT from the owner to MeltiFy) to be logged on the blockchain before initiating the lottery creation process. This is done to prevent failed transactions. If at any point the transaction fails, an error message is displayed to the user.

\subsubsection{Profile page}
The Profile page can only be accessed by a logged user. After logging in it will immediately show, in the top right corner, the balance in ChocoChips. In the middle of the page, two containers will show the owned lotteries and the lotteries for which the user has bought at least one WonkaBar.\\
\indent In this page, as in the Lottery page, lotteries are shown as cards in which the NFT image is the most important part. Card descriptions and buttons are different based on the state of the lottery.

\paragraph{Owned lotteries} 
The expire date is the first information shown (expressed in the browser's language format), followed by the number of WonkaBars sold over the total supply of WonkaBars, and the amount of ETH to repay. In this section, the amount to repay is shown in its entirety, no matter how small it is, in order to give the user an accurate overview of their lotteries' state.\\
Here the button can be used to repay the loan: when clicked, Metamask will ask the user to sign the transaction, in which the amount to repay is already set. In case an error occurs, an alert box is prompted.\\
An example transaction that uses this function can be found at \href{https://goerli.etherscan.io/tx/0xc5594de7a71523fe548bad3646cd0cbb49d2c1fbdf3b361ba0cb37db3ff1b984}{\texttt{0xc559...b984}}.

\paragraph{WonkaBars bought} 
Lotteries for which the user has bought at least one WonkaBar share some common information with the owned lotteries (expire date, WonkaBars sold over the total supply) but the button's function is completely different: here the button is used to melt all the owned WonkaBars to receive a reward, based on the state of the lottery. Also some additional information is shown:
\begin{itemize}
    \item \textbf{Active lotteries}: the state of the lottery (Active) and the win probability for the user: this value is computed as $wonkaBarsOwned / wonkaBarsSold * 100$ (rounded to the second decimal digit), since no matter how much the total supply of WonkaBars is, what counts is how many WonkaBars have been sold and how many of them are owned by the user.\\
    In this case the button is disabled, since WonkaBars can only be melted after the lottery is no more active.
    \item \textbf{Cancelled lotteries}: the state of the lottery (Cancelled) is followed by the winner, which in the case of a cancelled lottery is None. Instead of the win probability (useless now since the lottery is already over) the user is told what they will receive when melting WonkaBars: for cancelled lotteries the user will receive the refund and some ChocoChips.\\
    Clicking on the button will open a Metamask window in which the user will be asked to sign the transaction.\\
    An example transaction that uses this function can be found at \href{https://goerli.etherscan.io/tx/0xd957d276b5499f6ad40e62a47ad9fd74f255f0a6fc6e1240903fd8aaf1337c3f}{\texttt{0xd957...7c3f}}\\
    \item \textbf{Concluded lotteries}: the state of the lottery (Concluded) is now followed but the address of the winner. The address, too long to fit in a card, is shortened in the form \texttt{0x0123....4567} and is directly linked to the address page on Etherscan\cite{GerliEtherscan}. For lotteries in this state the user will only receive ChocoChips, since the loan has not been repaid.
\end{itemize}

\subsection{Deepening in the implementation}
A significant amount of time and effort was dedicated to carefully evaluating and selecting the most appropriate frameworks for our project. The rationale behind this approach is that investing time in researching and selecting the right frameworks can ultimately save time in the long run.
\\
\indent One such framework that proved to be worthwhile was React-Bootstrap. Despite its steep learning curve, it allowed us to create a visually pleasing website that would have been difficult to achieve using plain JavaScript alone. However, the documentation for this framework proved to be a challenge, requiring a significant amount of time to fully understand and utilize it effectively.
\\
\indent Unfortunately, not all frameworks were as successful. ethers-react, for example, promised to be highly compatible with React, but ultimately failed to function. For connecting to the backend, we ultimately chose to use the ethers.js library, which is known for its lightweight characteristics compared to web3. Despite initial difficulties in implementation, we were ultimately able to make it work.
\\
\indent We also explored the Thirdweb framework, which promised to simplify the development process. While it did simplify certain aspects, such as the login button, it ultimately proved to be ineffective for other aspects of the project. Despite initial efforts to make it work, we ultimately returned to using the Ethers framework for rendering NFTs, which proved to be a slower (but functioning) process.

\subsection{MeltyFi.DAO DApp}
Although the MeltyFiDAO smart contract has been successfully deployed on the Goerli Testnet, we have not yet implemented a user interface to facilitate interactions with the contract. This was a conscious decision as we wanted to provide a foundation for developers to build upon, leaving the MeltyFiDAO contract as a skeleton structure for future development and customization.
While the user interface for interacting with the MeltyFiDAO smart contract has not yet been developed, it is still possible for users to interect with the contract through the use of \href{https://goerli.etherscan.io/address/0xC4AA65a48fd317070F1A5aC5eBAC70F9d022Fb1e#writeContract}{Goerli Etherscan interface}.