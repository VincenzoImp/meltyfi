\documentclass[main.tex]{subfiles}

\begin{document}

\section{Introduction}\label{sec:introduction}


blockchain technology 
    

NFT 
    definition. too volatility, 
    examples
    why are they valuable
    
lending and borrowing platform
    collateral
    liquidation
    pool 
    
loan with nfts as collateral 
    peer to peer
        definition
        pro 
        con
        existings protocols
    peer to pool
        definition
        pro
        con
        existings protocols
    limitations

meltify
    definition
    problem solved
    limitation
    


cosa sono gli nft
Non-fungible tokens (NFTs) have become popular as unique and non-interchangeable units of
data that signify ownership of associated digital items, such as images, music, or videos. Token
“ownership” is recorded and tracked on a blockchain \cite{CRS_NFT}. 

bayc e' una perfetta manifestazione del fenomeno

a causa di molti fattori, e' difficile decifrare il vero valore e quindi definire il giusto prezzo.
il floor price è un indicatore di prezzo abbastanza onesto \cite{FloorPrice}

questo però e' un fattore offchain e quindi è soggetto a maggior ragione a tante cose che possono alterare, contaminare il corretto valore: volatilita, manipolazioni di mercato, oracle manipulation, e tutti gli attacks che possono essere effettuati durante il meccanismo di consenso di ethereum (sandwitch attack ecc)

a causa di molti fattori, è difficile liquidare gli nft 

utilità degli nft:
come status symbol, profile picture, tanti altri use case attuali

uno degli ultimi use case è quello di usare nft in defi

la defi sono i protocolli di finanza decentralizzati presenti su ethereun e quindi in blockchain
questi tipicamente permettono lo swap tra vari asset o il lending e borrowing

vegono quindi usati come colaterale per prestiti, ma a causa dei due problemi principali:l'illiquidità è l'incertezza del loro valore/floor price, è difficile impiegarli come collaterale e occorre fare molti compromessi. loan ratio basso, overcollateralizzazione, floor price dropping, liquidation, illiquidità in acquisto di un nft liquidato, e tanto altro ancora.

spiegazione dei protocolli vigenti e tutte le loro problematiche o rislosuzioni
peer to peer e peer to pool

notiamo che i due problemi principali sono prima di tutto la liquidazione che dipende da fattori di mercato offchain con tutti i possibili attacchi e instabilità oltre a tutti i problemi che seguono la liquidazione come illiquidità in acquisto di un nft liquidato,
se si evita questo problema la soluzione sarebbe organizzare un prestito p2p accordandosi su un prezzo e su una deadline e sugli interessi
questo debella il primo problema ma evince la difficolta di trovare un accordo tra due persone e ad aggravare questo occorre anche che il lender abbia tutto il capitale da prestare e che a fronte di una non restituzuone di questo, il lender sia interessato ad ottenere l'nft a collaterale

spiegazione di meltyfi dettagliata 

meltyfi è un protocollo di prestiti collateralizzati con nft con i benefici del peer to pool e i benefici di p2p evitando tutti i problemi di queste due modalità di lending/borrowing. aggiungendo solo pochissimissimi compromessi per il lender (possibilità di avere il biglietto perdente e perdere dal suo investimento, ma questo dipende da una scelta del lender) per il borroware (non possibilita di avere un flash load, ma uno slow loan nel vendere i biglietti, ma questo dipende dalla lotteria che il borroware organizza e dalla sua bravura di vendere biglietti)

a fronte di compromessi che gli agenti sono disposti personalmente a prendere e tollerare, il sistema funzione e da benefici a tutti rendendo efficiente e molto vantaggioso per tutti quello che prima era difficile da ottenere

ps ricordarsi di confrontare peer2peer e peer2pool e meltyfi riassumendo tutti i pro e contro








architettura di meltyfi

token

ticket

possibili problematiche 



\end{document}


motivare che la nostra dapp non e' una piattaforma per pubblicare lotterie con il riservo di annullarle prima della scadenza

meltyfi non e' semplicemente quanto detto sopra perche

mostrare il problema dell'oracle manipulation
mostrare il fatto che tutti i protocolli di loan backed on nfts as collateral sono soggetti ad attacchi esterni di vario tipo. come Oracle manipulation, floor price problem, imposibile determinare il bene 

soluzione senza problemi si base sull'accordo di due persone con scadenza temporale
ma occorre il capitale dell'una per l'altra

soluzione far mettere il capitale a piu' persone 
in caso di inadempienza? a chi dare l'nft 
soluzione... lotteria
incentivi e cosi via

importantissimo evidenziare come per prestiti collateralizzati con nft presentanio il problema delle soglia di liquidazione

cio comporta una ipercollateralizzazione per prevenire liquidazioni indesiderate


sottolineare anche la modalita' di liquidazione di un nft
in base alla rapidita e al prezzo




permettiamo loand backed on nfts collateral











MeltyFi.NFT


- GitHub repo
https://github.com/VincenzoImp/MeltyFi.NFT

- Report
https://it.overleaf.com/2444962516sxzpmdrtqhgz

- Meltyfi frontend
https://meltyfi.nft
https://meltyfi.dao

- Zoom room
https://uniroma1.zoom.us/j/85419901259?pwd=bjNGVmNSWFYwU3FkaWE1K1c2bFBSQT09

- Theme inspiration
https://chocofinance.com

- Similar Dapps
https://drops.co
https://liquidnfts.com
https://www.nftfi.com

- Setup and practice
https://remix-project.org
https://metamask.io
https://goerlifaucet.com
https://testnets.opensea.io
https://goerli.etherscan.io

- Resources
https://youtu.be/gyMwXuJrbJQ
https://github.com/smartcontractkit/full-blockchain-solidity-course-js
https://solidity-by-example.org
https://docs.soliditylang.org/en/v0.8.17
https://www.tutorialspoint.com/solidity
https://ethereum.stackexchange.com
https://github.com/OpenZeppelin/openzeppelin-contracts